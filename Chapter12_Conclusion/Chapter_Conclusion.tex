\section{Summary of contributions}

The contributions in this thesis were organized into three distinct parts, with each part focusing on a specific challenge relating to the use of Volterra series models in system identification and control. 

\subsection{Part \ref{part:TD} - Time domain identification}

Contributions in this part of the thesis were aimed at improving the feasibility of high order Volterra series identification in the time domain under arbitrary experimental conditions. For this particular challenge, the achieved results were as follows.

In Chapter \ref{chap:4}, an alternate algorithm was developed for hyperparameter optimization in regularized Volterra series estimation. The proposed method uses an expectation-maximization (EM) framework to split the total optimization problem into smaller subproblems in an iterative scheme. When compared to the often used marginal likelihood maximization (MLM) approach, the EM method features better computation time scaling with Volterra series order, which was demonstrated through a Monte Carlo simulation study.

Chapter \ref{chap:5} proposed the use of Bayesian regularization techniques directly on basis function expansions of Volterra kernels. This new concept, which is supported both theoretically and by linear identification literature, combines the parameter reduction benefits of basis function expansions with the accuracy benefits of regularized estimation. An extension to the EM method of Chapter \ref{chap:4} was provided for the basis function case, whereby the covariance hyperparameters and the basis generating parameters can be simultaneously optimized in an iterative fashion. The regularized basis function approach was shown, through numerical examples, to enable feasible model estimation at much higher Volterra series orders than was previously possible with existing methods.

In Chapter \ref{chap:6}, further evidence was provided for the performance of regularized basis function estimation of Volterra series models. The method proposed in Chapter \ref{chap:5} was applied to two real datasets, sourced from nonlinear benchmark systems in the system identification community. Basis function models obtained for the first system, a set of vertically cascaded tanks, were shown to provide equivalent prediction accuracy to that of an equivalent time domain model, while the computation times for the estimates were orders of magnitude lower. For the coupled electric drives benchmark, a second order model obtained using this thesis' proposal was shown to outperform a number of popular nonlinear estimation routines. 

\subsection{Part \ref{part:FD} - Frequency domain identification}

This part of the thesis considered Volterra series identification in the frequency domain, where there existed no equivalent Bayesian estimation methods to those available in the time domain. The contributions were organized into three chapters as follows.

Chapter \ref{chap:7} focussed on estimation using the NOFRF model, and developed a Gaussian process regression (GPR) method for the simultaneous identification of all NOFRFs from a single experiment. The scope of the method was limited to parallel Hammerstein systems, in which case the NOFRFs can be guaranteed to behave like input-independent linear filters. The proposed method was shown through numerical examples to significantly outperform a traditional multilevel excitation approach, while using less data and requiring fewer experimental restrictions.

In Chapter \ref{chap:8}, a framework was developed for the direct estimation of generalized frequency response functions (GFRFs) using Bayesian priors. By appropriately transforming the prior covariance structures designed for time domain Volterra series estimation, an equivalent frequency domain method to that of \cite{Birpoutsoukis2017} was constructed. The proposed GPR method was shown to produce accurate estimates under rank deficient conditions, and a Monte Carlo study demonstrated the superiority of the direct frequency domain approach in the case where a limited frequency band is of interest.

Chapter \ref{chap:9} explored the output response, in the frequency domain, of nonlinear Volterra systems when they have arbitrary excitation, i.e. they are not in steady-state. Expressions were derived in the chapter for the steady-state and transient contributions to the output spectrum at each nonlinear order. The structure of the transient expressions provided insight which is relevant to frequency domain system identification. In particular, the transients at higher nonlinear orders have a more complex structure due to their increased dependence on the measured input sequence, and this behaviour violates the assumptions of many popular linear methods for transient estimation and removal. 

\subsection{Part \ref{part:MPC} - Model predictive control}

The final part of the thesis investigated the use of Volterra series models in a control context. Historically, such models could only be used for control if they were restricted to very low orders with a simplified structure. The results obtained for this control-related challenge were as follows.

In Chapter \ref{chap:10}, a model predictive control (MPC) algorithm was constructed for arbitrary order Volterra series models expressed using Laguerre basis function expansions. Using a new Kronecker algebra framework for the nonlinear state space model, the MPC optimization problem at each time step was shown to reduce to a polynomial root finding problem, which can be solved in a computationally efficient manner. The presence of input constraints and output/model disturbances were also addressed through algorithm extensions.

Chapter \ref{chap:11} provided two application examples for the control method designed in Chapter \ref{chap:10}. The prediction models for each system were obtained once again using the regularized basis function approach proposed in Chapter \ref{chap:5}. Control performance for each example was compared with MPC using simplified and linearized models, with positive results.

\section{Considerations for future work}

For the topics discussed in this thesis, there are numerous potential pathways for future research. Some of the promising extensions are identified below.  

In the area of time domain identification, there is more to explore with regards to the use of regularized basis function expansions. In Chapter \ref{chap:5}, Laguerre and Kautz bases were discussed as two simple but popular orthonormal sets, however there are many other choices for the basis functions which may be advantageous in different contexts. Wavelets in particular have received significant attention already in the literature for Volterra series estimation.

There are also several avenues for analyzing and extending the EM hyperparameter tuning methods in Algorithms \ref{alg:EMtuning} and \ref{alg:BFopt}. Further analysis can be performed on the modified EM Algorithm \ref{alg:BFopt}, to determine the convergence properties of the basis function parameters within the scheme. To increase computational efficiency, it may be useful to look at accelerated EM techniques \cite{McLachlan2007}. There are also two layers of parallel computing potential in the EM algorithms. The first layer comes from the optimization problem in each iteration being partitioned into independent subproblems for each nonlinear order. The second layer resides in the individual non-convex optimizations, which can be performed in a parallel fashion using for example a particle swarm approach \cite{Kennedy1995}, \cite{Eberhart1995}. 

Recursive identification schemes were not considered within this thesis, however the iterative nature of the EM-based methods in Chapters \ref{chap:4} and \ref{chap:5} make them good candidates for conversion to a recursive formulation. Such a formulation would be particularly useful in tracking the dynamics of nonlinear systems with a time-varying nature.

Moving now to the frequency domain setting, there are several open questions remaining. The results in Chapter \ref{chap:8} lead to an equivalent frequency domain method to that of \cite{Birpoutsoukis2017}, however the method only applies for periodic and steady-state estimation data. It is possible to relax this constraint by incorporating the transient behaviour, described in Chapter \ref{chap:9}, into the GPR method. Future work would investigate how best to express the transient behaviour in a Bayesian setting while still maintaining computational feasibility. 

There are also opportunities for further analysis of the transient expressions in Chapter \ref{chap:9}. While it has been shown that the transients at high nonlinear orders are structurally more complex than linear transients, they can still be locally smooth (in the frequency domain) if the sample size is sufficiently large with respect to the memory length of the system. This is because a larger sample size implies higher frequency resolution in the output DFT. Thus, another question for future research is whether methods such as LPM and LRM can still be used to remove the transient function when using very large sample sizes.

A final important question for frequency domain identification is whether a regularized basis function approach can also be used in this domain. The orthonormal basis functions considered in Chapter \ref{chap:5} possess frequency domain expressions, and so it should be possible to use these or other appropriate bases to directly estimate a compact basis function model using Bayesian regularization, equivalently to the time domain case.

With respect to the MPC contributions in Chapter \ref{chap:10}, several improvements can be made to enrich the proposed control method. First, the state space framework can be adapted to accommodate different basis function expansions, particularly the 2-parameter Kautz case which was shown in Chapter \ref{chap:5} to better capture resonant dynamics. Incorporating the Kautz filters into a state space model is possible (see e.g. \cite{Mbarek2003}), however the structure of the dynamic equation will be different to the Laguerre filter case. 

Removing the move horizon restriction, $\mu=1$, should also be considered in future research. There has been some progress towards this goal in \cite{Parker2002}, which looked specifically at second order Volterra-Laguerre models. The MPC was implemented for larger move horizons by using a locally linearized plant model for optimizing all but the first sample of the optimal input sequence. 

Finally, for arbitrary order models, the explicit theoretical requirements for stability of Algorithm \ref{alg:BasicVL-MPC} have not been established. It may be possible to obtain such results using Lyapunov theory, or by bounding the sequence of objective function values by an appropriate monotonically decreasing function. There may also be additions or modifications to the algorithm which can help analyze and guarantee its stability without forfeiting the root-finding nature of the optimization problem.